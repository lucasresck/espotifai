\documentclass{article}

\usepackage[utf8]{inputenc}
\usepackage{authblk}
\usepackage{hyperref}
\usepackage[margin=3cm]{geometry}
\usepackage{listings}
\lstset{
  language=bash,
  basicstyle=\ttfamily
}

\title{Espotifai\\
\vspace{10pt}
\large Automatic Playlist Recommender}
\author[]{Lucas Emanuel Resck Domingues}
\author[]{Lucas Machado Moschen}

\affil[]{\textit{School of Applied Mathematics}
\\ \textit{Getulio Vargas Foundation}}

\date{\today}

\begin{document}

\maketitle

\begin{center}
    \href{https://github.com/lucasresck/espotifai}{GitHub Repository}
\end{center}

\section{Project statement}

    \textbf{Question}: Which song should we recommend based on
    a playlist and user or context information?

    We can define playlist as a sequence of tracks (audio recordings).
    In this project we aim to study the playlist generation problem, that is,
    given a pool of tracks, a background knowledge about the user,
    and some metadata of songs and playlists, the goal is to create a sequence
    of tracks that satisfies some target as best as possible. The notebooks,
    scripts and work can be found in our repository.

\section{The Datasets}

In order to obtain user information, we generated a list of usernames in a
networked way.  We visited the Last.fm webpage of several artists and
considered three random users in the top listenings from three different
coutries: Brazil, USA and United Kingdom. Using only these usernames, through
the command \lstinline{python generate_lastfm_users.py},  we get additional
Last.fm usernames using the \lstinline{user.getFriends} method. 
With some loops, we can get the network (or part of it). It's possible to have
some bias, unknown yet.

\subsection{Spotify Database}

\subsection{Last.fm Database}

\section{Gained Insights}

\section{Baseline Model}

\end{document}